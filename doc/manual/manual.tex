\documentclass[a4paper,11pt]{scrartcl}

\usepackage[utf8]{inputenc}
\usepackage{amsmath}
% \usepackage{makeidx}
\usepackage[T1]{fontenc}

\usepackage{ifpdf}
\ifpdf
    \usepackage[pdftex]{graphicx}
    \usepackage[pdftex]{hyperref}
\else
     \usepackage{graphicx}
\fi
% some stuff to make life easier
\newcommand{\pymvpa}{PyMVPA~}
\newcommand{\code}[1]{\texttt{#1}}

\author{Michael Hanke}
\title{\pymvpa -- Multivariate Pattern Analysis with Python}
\date{\today}

% \makeindex

\begin{document}

\maketitle
\begin{abstract}
Write a nice little summary about \pymvpa\ldots
\end{abstract}

\begin{center}
This manual has been last updated for \pymvpa version xxx.
\end{center}

\tableofcontents

\section{What MVPA is good for?}



\section{A short example}



\section{Classifiers}
\subsection{Support vector machines}
\subsection{k-nearest neighbours}
\subsection{Logistic regression}



\section{Feature selection strategies}
\subsection{Searchlight}
\subsection{Recursive feature elimination}
\subsection{Incremental feature search}



\section{Notes for developers}
\subsection{Unit tests}
Every part of \pymvpa should have a reasonable set of tests that ensure it is
working as expected. All unit tests have to be put into the \code{tests}
directory and shall be registered in the \code{tests/tests\_main.py} file.

\subsection{Requirements for classes providing classifiers}
With the \code{CrossValidation} class \pymvpa provides a generic cross-validation
algorithm than can make use of an arbitrary classifier. \pymvpa already provides
a number of algorithms, but additional classifiers can easily be added. There
are only a few requirements that such classifier class has to comply with:

\begin{description}
 \item[One required constructor argument] The constructor must not have more
    than one required argument. This argument is a \code{MVPAPattern} object
    containing the training data -- and only the training data. The constructor
    may take an arbitrary number of additional \textit{keyword arguments}.
 \item[Automatic training] When a classifier object is created, the classifier
    must automatically train itself. There must not be the need to make
    additional method calls, e.g. the constructor has to deal with possible
    problems like convergence errors automatically or raise an exception if it
    cannot handle it.
 \item[Must have \code{predict()} methods] Each classifier class has to provide
    a \code{predict()} method that can be used to retrieve the classifiers
    prediction for some test data. That method has to accept a sequence with the
    test data and may not require more than this argument. The method has to
    return a sequence with the list of predicted regressors -- one prediction
    for each element of the supplied testdata sequence.
 \end{description}

In addition classifiers may provide more functionality like the
\code{rateFeatures()} method required by the recursive feature elimination
algorithm. However, this is not strictly necessary.

\subsection{Passing arguments to the \code{CrossValidation} class}



% Put the index at the end
% \index{Test}
% \printindex
\end{document}
