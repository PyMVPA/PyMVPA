\documentclass[a4paper,11pt]{article}
\usepackage[latin1]{inputenc}
\usepackage[english]{babel}
\usepackage{epsfig}
\usepackage{amsmath}
\usepackage{amsfonts}

\newcommand\R{{\mathbb R}}
\newcommand\x{{\mathbf x}}
\newcommand\X{{\mathbf X}}
\newcommand\K{{\mathbf K}}
\newcommand\J{{\mathbf J}}
\newcommand\LL{{\mathbf L}}
%%\newcommand\L{{\mathbf L}}
%%\DeclareMathOperator*{\argmax}{arg\,max}
%% \DeclareMathOperator*{\argmax}{argmax} %% (argmax wihtouth mid space)
\DeclareMathOperator*{\argmin}{arg\,min}
\DeclareMathOperator*{\var}{var}
\newcommand{\Rvec}[1]{{\bf #1}}
\newcommand{\Ivec}[1]{\mbox{\boldmath $#1$}}

\title{Kernels}
\author{Emanuele Olivetti}

\begin{document}

\maketitle

\section{Introduction}
This document gives a detailed description of kernels implemented in
PyMVPA together with derivation of their gradients. Gradients are
useful when computing and maximizing the log marginal likelihood of a
Gaussian process approximating data.

The following notation is be used:
\begin{itemize}
\item $\x \in \R^D$ : a $D$-dimensional column vector, $\x =
  (x_1,\ldots,x_D)$.
\item $\X = (\x_1^{\top},\ldots,\x_N^{\top})$ : a $N \times D$
  matrix where each row is a $D$-dimensional vector. $\X$ is also
  called set of \emph{samples}. $\X_{* i}$ indicates the $i$-th
  column of $\X$ (column vector), and $\X_{j *}$ indicates the
  $j$-th row of $\X$ (row vector).
\item $k: \R^D \times \R^D \rightarrow \R$ : a covariance function.
\item $\K(\X,\X')$ : the matrix extension of $k$.\, i.e., $\K_{pq} =
  k(\x_p,\x'_q)$.
\item $\J_{n,m}$ : the $n \times m$ matrix of ones, i.e., a matrix
  where each element is 1. When $n = m$ it can be denoted as $\J_n$.
\item $\|\mathbf{z}\|$ : the norm of vector $\mathbf{z}$. In case of
  Euclidean norm $\|\mathbf{z}\| =
  \sqrt{\mathbf{z}^{\top}\mathbf{z}}$.
\item $\|\X-\X'\|$ : the \emph{distance matrix} between $\X$ and $\X'$
  defined element by element as $\|\X-\X'\|_{pq} = \|\X_{p *} -
  \X'_{q *}\|$.
\item $\X \bullet \mathbf{Y}$ : the Hadamard (or Schur) matrix
  product, i.e. the entrywise product between matrices of the same
  size. Let $\mathbf{Z} = \X \bullet \mathbf{Y}$, then $z_{ij} =
  x_{ij} y_{ij}$.
\end{itemize}

\section{Constant kernel}
$$k(\x,\x') = \sigma_0^2$$
where $\sigma_0 \ge 0$ is the standard deviation of the Gaussian prior
probability $\mathcal{N}(0,\sigma_0^2)$ of the value of the constant.
$$\K(\X,\X') = \sigma_0^2 \J_{n,m}$$
$$\mathbf{\Theta} = \{\sigma_0\}$$
$$\frac{\partial k}{\partial \sigma_0}(\x,\x') = 2\sigma_0$$
$$\frac{\partial \K}{\partial \sigma_0} = 2\sigma_0 \J_{n,m}$$
$$A = \sigma_0^2$$
$$A \ge 0$$
$$k(\x,\x') = A$$
$$\K(\X,\X') = A \J_{n,m}$$
$$\frac{\partial k}{\partial A} = 1$$
$$\nabla \K_A = \frac{\partial \K}{\partial A} = \J_{n,m}$$

\section{Linear kernel}
Let $\Ivec{\Sigma}_p$ be the $D \times D$ covariance matrix of the Gaussian
prior probability $\mathcal{N}(\Ivec{0},\Ivec{\Sigma}_p)$ of the weights of
the Bayesian linear regression.
$$k(\x,\x') = \x^{\top} \Ivec{\Sigma}_p \x'$$
$$\K(\X,\X') = \X \Ivec{\Sigma}_p \X'^{\top}$$
In order to simplify formulas we assume $\Ivec{\Sigma}_p$ is diagonal, i.e.,
$\Ivec{\Sigma}_p = \Ivec{\sigma}^2_p I$ where $\Ivec{\sigma}^2_p =
\{{\sigma^2_p}_1,\ldots,{\sigma^2_p}_D\}$:
$$k(\x,\x') = \sum_{i=1}^D {\sigma^2_p}_i x_i x'_i$$
$$\mathbf{\Theta} = \{{\sigma_p}_1,\ldots,{\sigma_p}_D\}$$
$$\frac{\partial k}{\partial {\sigma_p}_i} = 2 {\sigma_p}_i x_i x'_i$$
$$A_i = {\sigma_p^2}_i$$
$$A_i \ge 0$$
$$\mathbf{A} = (A_1,\ldots,A_D)^{\top}$$
$$k(\x,\x') = \x^{\top} \mathbf{A}I \x'$$
$$\K(\X,\X') = \X \mathbf{A}I \X'^{\top}$$
$$\frac{\partial k}{\partial A_i} = x_i x'_i$$
$$\frac{\partial \K}{\partial A_i} = \X_{* i} {\X'_{* i}}^{\top}$$
$$\nabla_{\mathbf{A}} \K = ( \X_{* 1} {\X'_{* 1}}^{\top}, \ldots, \X_{* D} {\X'_{* D}}^{\top})$$
As expected the gradient is independent of the hyperparameters values
and can be computed one at the beginning.

\section{Polynomial kernel}
$k(\x,\x') = (\x^{\top} \Ivec{\Sigma}_p \x')^p$

\section{Exponential kernel}
\subsection{Scalar Lengthscale $\ell$}
$$k(\x,\x') = \sigma_f^2 e^{-\frac{\|\x-\x'\|}{2\ell}}$$
$$\ell \ge 0$$
$$\sigma_f \ge 0$$
$$\K(\X,\X') = \sigma_f^2 e^{-\frac{1}{2\ell}\|\X-\X'\|}$$
$$A = \sigma_f^2$$
$$A \ge 0$$
$$B = -\frac{1}{2\ell}$$
$$B \le 0$$
$$k(\x,\x') = A e^{B\|\x-\x'\|}$$
$$\K(\X,\X') = A e^{B\|\X-\X'\|}$$
$$\frac{\partial k}{\partial A} = e^{B\|\x-\x'\|} = \frac{1}{A}k(\x,\x')$$
$$\frac{\partial \K}{\partial A} = e^{B\|\X-\X'\|} = \frac{1}{A} \K(\X,\X')$$
$$\frac{\partial k}{\partial B} = \|\x-\x'\| A e^{B\|\x-\x'\|} =
\|\x-\x'\| k(\x,\x')$$
$$\frac{\partial \K}{\partial B} = \|\X-\X'\| \bullet \K(\X,\X')$$
$$\nabla_{A,B} \K = (\frac{1}{A} \K(\X,\X'), \|\X-\X'\| \bullet
\K(\X,\X'))$$

Note that when $\K(\X,\X')$ is already computed, the gradient requires
just two element-by-element products to be computed, the second being
against a constant matrix independent of the hyperparameters. So a
mechanism that stores the precomputed kernel matrix related to a given
gradient will reduce greatly the number of computations
(memoization?).

\subsection{Vector of Lengthscales $\Ivec{\ell}$}
Given $\Ivec{\ell} = (\ell_1,\ldots,\ell_D)$, $\ell_i \ge 0$, and $\LL
= \Ivec{\ell}I$
$$k(\x,\x') = \sigma_f^2 e^{-\frac{1}{2}\|\x-{\LL^{-1}\x'}\|}$$
$$K(\X,\X') = \sigma_f^2 e^{-\frac{(\x-\x')^{\top} \LL^{-1} (\x-\x')}{2}}$$


\section{Squared Exponential kernel}
$$k(\x,\x') = \sigma_f^2 e^{-\frac{1}{2}(\x-\x')^{\top} \LL^{-1} (\x-\x')}$$

\section{$\gamma$-Exponential kernels}

\section{Mat\'ern kernels}

\section{Rational Quadratic kernels}

\end{document}
